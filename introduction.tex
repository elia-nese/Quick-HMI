Ce travail de fin d'étude porte sur le développement d'une interface utilisateur homme-machine configurable pour des machines de type CNC, dans le but d'homogénéiser les interfaces des différentes machines au sein d'une entreprise.

Les machines de type CNC sont largement utilisées dans différents secteurs industriels pour effectuer des opérations de fabrication complexes et précises. Cependant, chaque machine peut avoir une interface utilisateur différente, ce qui peut causer des difficultés pour les opérateurs qui travaillent avec plusieurs machines de différents constructeurs au sein d'une entreprise.

L'objectif de ce travail est de développer une interface utilisateur flexible et configurable, permettant aux administrateurs système de personnaliser l'interface en fonction des besoins spécifiques de l'entreprise et des opérateurs et d'uniformiser l'architecture ainsi que le style des interfaces des différentes machines. Pour atteindre cet objectif, nous allons nous appuyer sur le framework Concept.Convergence, développé par l'entreprise Objectis pour C\# sur Visual Studio.


Ce travail s'articulera autour de plusieurs étapes clés, comprenant notamment l'étude de la technologie développée par Objectis, la mise en place de la configuration de l'interface, la communication avec des automates de constructeurs différents ainsi que le développement de l'application finale.
\section{Contexte}
L'entreprise Objectis a développé une technologie qui se nomme Concept afin de développer des applications graphiques standardisées.

\section{Citations et bibliographie}
Documentation interne de l'entreprise Objectis.

