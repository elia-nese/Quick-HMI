\documentclass[
    iai, % Saisir le nom de l'institut rattaché
    eai, % Saisir le nom de l'orientation
     % Décommentez si le travail est confidentiel
]{heig-tb}

\usepackage[nooldvoltagedirection,european,americaninductors]{circuitikz}
\usepackage{graphicx}
\usepackage{hyperref}
\usepackage[table,xcdraw]{xcolor}

\signature{elianese.svg} % Remplacer par votre propre signature vectorielle.

\makenomenclature
\makenoidxglossaries
\makeindex

\addbibresource{bibliography.bib}

\input{nomenclature}
\input{acronyms}
\input{glossary}
% Auteur du document (étudiant-e) en projet de Bachelor
\author{Elia Nese}

% Activer l'option pour l'accord du féminin dans le texte
\genre{male}

% Titre de votre travail de Bachelor
\title{Quick-HMI}

% Le sous titre est optionnel
\subtitle{Travail de Bachelor}

% Nom du professeur responsable
\teacher {Prof. F. Birling (HEIG-VD)}

% Mettre à jour avec la date de rendu du travail
\date{\today}

% Numéro de TB
\thesis{7212}



\surroundwithmdframed{minted}

%% Début du document
\begin{document}
\selectlanguage{french}
\maketitle
\frontmatter
\clearemptydoublepage

%% Requis par les dispositions générales des travaux de Bachelor
\preamble
\authentification

%% Résumé / Résumé publiable / Version abrégée
\begin{abstract}
    \input{abstract}
\end{abstract}

%% Sommaire et tables
\clearemptydoublepage
{
    \tableofcontents
    \let\cleardoublepage\clearpage
    \listoffigures
    \let\cleardoublepage\clearpage
    \listoftables
    \let\cleardoublepage\clearpage
    \listoflistings
}

\printnomenclature
\clearemptydoublepage
\pagenumbering{arabic}

%% Contenu
\mainmatter
\chapter{Introduction}
Ce travail de fin d'étude porte sur le développement d'une interface utilisateur homme-machine configurable pour des machines de type CNC, dans le but d'homogénéiser les interfaces des différentes machines au sein d'une entreprise.

Les machines de type CNC sont largement utilisées dans différents secteurs industriels pour effectuer des opérations de fabrication complexes et précises. Cependant, chaque machine peut avoir une interface utilisateur différente, ce qui peut causer des difficultés pour les opérateurs qui travaillent avec plusieurs machines de différents constructeurs au sein d'une entreprise.

L'objectif de ce travail est de développer une interface utilisateur flexible et configurable, permettant aux administrateurs système de personnaliser l'interface en fonction des besoins spécifiques de l'entreprise et des opérateurs et d'uniformiser l'architecture ainsi que le style des interfaces des différentes machines. Pour atteindre cet objectif, nous allons nous appuyer sur le framework Concept.Convergence, développé par l'entreprise Objectis pour C\# sur Visual Studio.


Ce travail s'articulera autour de plusieurs étapes clés, comprenant notamment l'étude de la technologie développée par Objectis, la mise en place de la configuration de l'interface, la communication avec des automates de constructeurs différents ainsi que le développement de l'application finale.
\section{Contexte}
L'entreprise Objectis a développé une technologie qui se nomme Concept afin de développer des applications graphiques standardisées.

\section{Citations et bibliographie}
Documentation interne de l'entreprise Objectis.



\chapter{Pré-étude}
\input{Pre_etude.tex}

\chapter{Développement}
\input{Developpement.tex}

\chapter{Planification}
\input{Planification.tex}

\chapter{Conclusion}
\input{conclusion.tex}

\clearpage
\printbibliography

\appendix
\appendixpage
\addappheadtotoc

%%if
\chapter{Première annexe}

Les annexes n'ont pas un contenu \underline{normatif} mais \underline{descriptif}. Tout contenu annexé ne doit pas être nécessaire à la bonne compréhension du travail.

Les annexes contiennent généralement :

\begin{itemize}
    \item les dessins mécaniques (mises en plan);
    \item les schémas électriques détaillés;
    \item des photographies du projet;
    \item des scripts et des extraits de code source;
    \item des documents techniques \pex \emph{datasheet};
    \item des développements mathématiques.
\end{itemize}
\section{Sous section}
\lipsum[1]
%%fi

\let\cleardoublepage\clearpage
\backmatter

\label{glossaire}
\printnoidxglossary
\label{index}
\printindex

% Le colophon est le dernier élément d'un document qui contient des notes de l'auteur concernant la mise en page et l'édition du document : il est parfaitement optionnel.
\input{colophon.tex}

\end{document}
